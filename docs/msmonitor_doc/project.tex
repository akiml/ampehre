%
% project.tex
%
% Copyright (C) 2015, Achim Lösch <achim.loesch@upb.de>, Christoph Knorr <cknorr@mail.uni-paderborn.de>
% All rights reserved.
%
% This documentation may be modified and distributed under the terms
% of the BSD license. See the LICENSE file for details.
%
% encoding: UTF-8
% tab size: 4
%
% author: Achim Lösch (achim.loesch@upb.de)
% created: 7/24/14
% version: 0.5.8 - change project name to ampehre
%

\section{Project Description}
The Ampehre project is a BSD-licensed modular software framework used to sample various types of sensors embedded in integrated circuits or on circuit boards deployed to servers with a focus to heterogeneous computing. It enables accurate measurements of power, energy, temperature, and device utilization for computing resources such as CPUs (Central Processing Unit), GPUs (Graphics Processing Unit), FPGAs (Field Programmable Gate Array), and MICs (Many Integrated Core) as well as system-wide measuring via IPMI (Intelligent Platform Management Platform). For this, no dedicated measuring equipment such as DMMs (Digital Multimeter) is needed. We have implemented the software in a way that the influence of the measuring procedures running as a multi-threaded CPU task has a minimum impact to the overall CPU load. The modular design of the software facilitates the integration of new resources. Though it has been enabled to integrate new resources since version v0.5.1, the effort to do so is still quite high. Accordingly, our plans for the next releases are broader improvements on the resource integration as well as an extensive project review to stabilize the code base. Version 0.8.0 features a client/server implementation, which moves the graphical rendering as well as other computations from the heterogeneous node to a client. 
