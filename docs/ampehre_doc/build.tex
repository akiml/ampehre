%
% build.tex
%
% Copyright (C) 2015, Achim Lösch <achim.loesch@upb.de>, Christoph Knorr <cknorr@mail.uni-paderborn.de>
% All rights reserved.
%
% This documentation may be modified and distributed under the terms
% of the BSD license. See the LICENSE file for details.
%
% encoding: UTF-8
% tab size: 4
%
% author: Achim Lösch (achim.loesch@upb.de)
% created: 7/24/14
% version: 0.5.8 - change project name to ampehre
%

\section{Build instructions}

Required dependencies:
\begin{itemize}
    \item CMake
    \item gcc, g++
\end{itemize}

\noindent Optional dependencies:
\begin{itemize}
    \item NVML for Nvidia GPU support
    \item maxeleros for Maxeler FPGA support
    \item mpss-micmgmt for Intel Xeon Phi support
    \item cpufrequtils/cpupowerutils
    \item QT and QWT for GUI tools
\end{itemize}

\noindent Included dependencies:
\begin{itemize}
    \item PAPI
    \item cjson
\end{itemize}

\subsection{Cloning}

Clone the project to a local directory.

~

\verb+git clone --recurse-submodules https://github.com/akiml/ampehre+

~

\noindent This also clones the PAPI repository as a submodule to the \verb+papi/+ folder.

\subsection{Building}

Ampehre uses CMake as primary build system.
The Makefile is used for starting the build, installing and cleaning up.
Run \verb+make+ to start the build process into the subdirectory \verb+build/+.

The PAPI build process was not changed and is triggered from the CMake build script.
The called script can be found in the file \verb+papi/cmake_build.sh+.

Run \verb+make clean+ to clean the CMake and PAPI build folders.

Using \verb+make install+ you can install Ampehre to your system.

\subsection{Adjust build parameters}

It might be necessary to adjust certain build parameters.
Foremost you can change paths to \emph{gcc} and \emph{g++} inside the file \verb+Makefile+.
Next you can deactivate options for certain supported measurement sources at the top of the root \verb+CMakeLists.txt+.

You might have a newer version of cpufrequtils installed named cpupowerutils.
In this case adjust the linked library in \\\verb+libmeasure/cpu_intel_xeon_sandy/CMakeLists.txt+ or create a symlink to \verb+libcpupower.so+ called \verb+libcpufreq.so+ at the adequate position.
However, there is no guarantee for compatibility to newer versions.

In case the paths to optional dependencies for the NVML or Maxeler libraries can't be found, adjust the search paths in the \emph{find external libraries} section of the root CMake file \verb+CMakeLists.txt+.

Note that it is most certainly necessary to call \verb+make clean+ after changing build parameters to remove the CMake cache and generated Makefiles.

\subsection{Linux Kernel Modules}

The \emph{driver\_measure} module offers access to mainboard power readings through DELL IDRAC7 IPMI messages.
To build and install the \emph{driver\_measure} module run \verb+make+ and \verb+make install+ in \verb+driver/+.
There are folders with two versions: the module in \verb+driver+ offers IPMI and MSR measurements, while the module in \verb+driver_ipmi+ only supports IPMI.
The old \emph{libmeasurement} requires the module in \verb+driver+.

The \emph{amsr} module offers access to MSR values used to read energy, temperature and frequency values for Intel CPUs.
It is a modified version of the original \emph{msr} kernel module, extended with whitelisting of available registers.
The \emph{amsr} device nodes can be made available to normal users since only whitelisted registers are accessible.
To add registers to the whitelist modify the arrays at the top of the \verb+amsr.c+ file and rebuild.
To build the \emph{amsr} module run \verb+make+.
Note: \emph{msr\_safe} is a similar kernel module supported by the PAPI rapl component.

For both modules, make sure that the respective device files are accessible to users.
Adjust accessibility using \emph{chmod}.


