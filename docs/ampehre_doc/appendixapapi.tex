\section{libapapi definition file specification}
\label{appendixapapi}

\subsection{eventlist file}

Every line is interpreted as one event name.
Lines starting with '\#' are ignored.
Empty lines are ignored.

Example:

\begin{verbatim}
# rapl
rapl:::PACKAGE_ENERGY:PACKAGE0
rapl:::PACKAGE_ENERGY:PACKAGE1
rapl:::DRAM_ENERGY:PACKAGE0
rapl:::DRAM_ENERGY:PACKAGE1
\end{verbatim}

If the \verb+APAPI_EVENTLIST+ environment variable is defined and the rapl component is active, only the four specified events are measured.


\subsection{eventops file}

Every line is interpreted as one event definition.
Lines starting with '\#' are ignored.
Empty lines are ignored.
One definition consists of several fields delimited by a ','.
Fields should not contain unnecessary whitespace.
One definition consists of the following fields:

\begin{itemize}

\item event name
\item operation to use to compute value1 from value0 (last two values for value0 are called sample1 (last) and sample0 (previous))

    \begin{itemize}
    \item \verb+APAPI_OP1_SAMPLE_DIFF+
        \begin{itemize}
        \item value1 = sample1 - sample0
        \end{itemize}
    \item \verb+APAPI_OP1_SAMPLE1_MUL_DIFF_TIME+
        \begin{itemize}
        \item value1 = sample1 * (time1 - time0)
        \end{itemize}
    \item \verb+APAPI_OP1_AVG_SAMPLE_MUL_DIFF_TIME+
        \begin{itemize}
        \item    value1 = (sample0 + sample1) / 2.0 * (time1 - time0)
        \end{itemize}
    \item \verb+APAPI_OP1_DIV_DIFF_TIME+
        \begin{itemize}
        \item    value1 = sample1 / (time1 - time0)
        \end{itemize}
    \end{itemize}
\item statistics to compute for value0, single term or '|' delimited list of terms:
    \begin{itemize}
    \item \verb+APAPI_STAT_NO+ - no statistics
    \item \verb+APAPI_STAT_MIN+
    \item \verb+APAPI_STAT_MAX+
    \item \verb+APAPI_STAT_AVG+
    \item \verb+APAPI_STAT_ACC+
    \item \verb+APAPI_STAT_ALL+
    \end{itemize}
\item statistics to compute for value1
\item maximal raw counter value or 0 if counter is not accumulating
\item name for type of value0
\item unit for value0
\item factor to use for value0
\item name for type of value1
\item unit for value1
\item factor to use for value1
\end{itemize}

Example:

\begin{verbatim}
# this event is an accumulating event
# the maximal raw counter value is not zero
rapl:::PP0_ENERGY:PACKAGE1,APAPI_OP1_DIV_DIFF_TIME,APAPI_STAT_ACC,
APAPI_STAT_MIN|APAPI_STAT_MAX|APAPI_STAT_AVG,0x7fffffff8000,
energy,Ws,2147483648.0,power,W,2.147483648

# this event is not-accumulating
# the maximal raw counter value is zero
rapl:::THERM_STATUS:PACKAGE0:CPU0,APAPI_OP1_NOP,
APAPI_STAT_MIN|APAPI_STAT_MAX|APAPI_STAT_AVG,APAPI_STAT_NO,
0,temperature,C,1,-,-,1
\end{verbatim}

\emph{Note:} The line breaks in the example are due to the document layout. The original entries must not contain line breaks.

Entries from the default file and the file declared in the \verb+APAPI_EVENTOPS+ environment variable are merged.
Entries in the user specified file override entries in the default file.
