%
% usage.tex
%
% Copyright (C) 2015, Achim Lösch <achim.loesch@upb.de>, Christoph Knorr <cknorr@mail.uni-paderborn.de>
% All rights reserved.
%
% This documentation may be modified and distributed under the terms
% of the BSD license. See the LICENSE file for details.
%
% encoding: UTF-8
% tab size: 4
%
% author: Achim Lösch (achim.loesch@upb.de)
% created: 7/24/14
% version: 0.5.8 - change project name to ampehre
%

\section{Usage}

Ampehre provides several possibilities to execute measurements.
\begin{itemize}
\item Measurement of a complete program execution
\item Integration into your own program to measure a certain section
\item Live monitoring of the system including a graphical output
\end{itemize}

\subsection{Tools}

\subsubsection{hettime}
\emph{hettime} allows to measure the execution of a newly started application similar to \emph{time}.
For this hettime starts the measurement, executes the program and waits for its completion before stopping the measurement.
hettime has several parameters to configure the measurement, but it expects at least the path to the executable.
Using \verb+-a+ you can pass arguments to the measured program.

~

\verb+hettime [OPTIONS] -e EXECUTABLE [-a ARGS]+

~

By default measurements for the modules are performed at a sampling rate of 100 ms.
To adjust the sampling rate use the following parameters in ms:

~

\verb+-c SAMPLE_CPU+

\verb+-g SAMPLE_GPU+

\verb+-f SAMPLE_FPGA+

\verb+-m SAMPLE_MIC+

\verb+-s SAMPLE_SYS+

~

By default hettime uses the libmeasurement library for measurment.
To start hettime with the libapapi library use the \verb+LD_PRELOAD+ environment variable.

~

\verb+LD_PRELOAD=/path/to/libms_common_apapi.so hettime+

~

This allows to use additional configuration of the APAPI library.

\subsubsection{msmonitor}

To do a live monitoring of your system you can use the msmonitor gui tool.

\subsubsection{msmonitor\_cs}

To use msmonitor remotely start the msmonitor server program on the host you want to monitor, the default port is 2900:

\verb+msmonitor_server [-p PORT]+

Now you can start the client on a different host to receive the monitored data:

\verb+msmonitor_client+

In the settings menu you can define the target address and port to connect to the monitored server.

\subsection{Program integration}

To integrate the measurement tool in your own program you need to include the \verb+ms_measurement.h+ header file and link against the \verb+libms_common.so+ library.
Now you can make use of the measurement library to measure a certain section of you program.
The following listing shows the necessary steps to execute a measurement.

\begin{lstlisting}
// initialize the measurement library using the default parameters
MS_VERSION version = { .major = MS_MAJOR_VERSION,
                       .minor = MS_MINOR_VERSION,
                       .revision = MS_REVISION_VERSION };
MS_SYSTEM *ms = ms_init(&version, CPU_GOVERNOR_ONDEMAND,
                      2000000, 2500000, GPU_FREQUENCY_CUR, 
                      IPMI_SET_TIMEOUT, SKIP_PERIODIC,
                      VARIANT_FULL);

// allocate the data structure for one measurement
MS_LIST *m1 = ms_alloc_measurement(ms);

// Set timer for m1
// Measurements perform every (10ms/30ms)
ms_set_timer(m1, CPU,    0, 10000000, 10);
ms_set_timer(m1, GPU,    0, 10000000, 10);
ms_set_timer(m1, FPGA,   0, 30000000, 10);
ms_set_timer(m1, SYSTEM, 0, 30000000, 10);
ms_set_timer(m1, MIC,    0, 30000000, 10);

// setup the m1 data structure for the next measurement
ms_init_measurement(ms, m1, CPU | GPU | FPGA | SYSTEM | MIC);

// measure the section you want to analyze
ms_start_measurement(ms);

do_something();

ms_stop_measurement(ms);
ms_join_measurement(ms);
ms_fini_measurement(ms);

// now you can look at the results

// destroy the measurement data structure
ms_free_measurement(m1);

// finally shutdown the measurement library
ms_fini(ms);
\end{lstlisting}

To analyze the measurement results use the methods in the respective module headers.

\begin{lstlisting}
// method for dram energy consumption
// defined in the ms_cpu_intel_xeon_sandy.h header
printf("consumed energy of cpu 1 dram bank : %.2lf mWs\n",
           cpu_energy_total_dram(m, 1));
\end{lstlisting}

\subsection{APAPI configuration}

On invocation libapapi reads certain environment variables for configuration.
This might be especially handy when using hettime with libapapi.
The detailed file formats are described in appendix \ref{appendixapapi}.

\begin{itemize}
\item \verb+APAPI_CMPLIST+ allows to define which components are measured.
The default is \verb+rapl:nvml:maxeler:micknc:ipmi+
\item \verb+APAPI_EVENTLIST+ allows to define the list of events per component that are measured.
The default list \verb+default_eventlist.txt+ is included.
\item \verb+APAPI_EVENTOPS+ allows to define how the events are processed.
This is only needed when adding new events.
The default list \verb+default_eventops.txt+ is included.
\end{itemize}


