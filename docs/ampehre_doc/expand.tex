\section{Extend measurement capabilities}

To add additional indicators you need to modify several files:

\begin{itemize}
\item PAPI component
\item libapapi event definitions
\item libapapi value copying
\item libmeasurement data structures and methods
\end{itemize}

\subsection{PAPI component}
The PAPI library supports extension by defining a new component.
The components can be found in the \verb+papi/src/components+ directory and every component has its own folder.
If you want to create a completely new component you can find an example component in the \verb+example+ folder.
Every component creates an internal list of available events on initialization in \verb+_example_init_component+.
\verb+_example_update_control_state+ is called to maintain a struct that contains the chosen events.
If the read method \verb+_example_read+ is called the values of chosen events need to be read and copied to the output array.
In all of this steps the different events need to be handled and modified in case of extension, however available components contain additional support methods.

If a new component is created it needs to be added to the list of compiled components \verb+--with-components+ in the build script \verb+papi/cmake_build.sh+.

\subsection{libapapi event definitions}

Once the component is available in PAPI the event definitions for libapapi must be adjusted.
The detailed file formats are described in appendix \ref{appendixapapi}.

The file defined in the environment variable \verb+APAPI_EVENTLIST+ contains the list of events to be considered if a component is activated.
Simply add the new events to the list similar to the default file.
Since the specified file will override the default file you also need to copy the desired events from the default file into the specified eventlist file.

The file defined in the environment variable \verb+APAPI_EVENTOPS+ contains a list of parameters that specifiy how to handle event values.
Accumulating values are treated differently.
Add a new entry per event according to the specifications in the appendix.
Your specified file will be merged with the default definitions so no copying of default definitions is necessary, except to override a default definition.

Finally, to activate the component you need to declare the list of used components in the \verb+APAPI_CMPLIST+ environment variable.

\subsection{libmeasurement data structures and methods}

For every libmeasurement component exists a struct containing the desired variables and the accessing methods.

\subsection{libapapi value copying}

To copy the measured event values to the libmeasurement structs the \verb+libms_common_apapi+ wrapper copies the values from the internal PAPI arrays into the target data structures.
To minimize the overhead of event traversing a list of mappings between the PAPI arrays and the libmeasurement structs are created in the method \verb+__create_mapper+.
Follow the pattern of the existing entries of checking the current component and existence of the event and finally defining the four entries for the mapping:

\begin{itemize}
\item \verb+source+ -- pointer to the PAPI array entry
\item \verb+destination+ -- pointer to the struct variable
\item \verb+type+ -- type conversion specifier
\item \verb+factor+ -- factor to convert value to the destination prefix
\end{itemize}

For struct variables that only need to be changed at the end of the measurement, check the \verb+last_measurement+ variable.

